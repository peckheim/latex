\section{Related Work}
Several previous studies investigated the use of optics in HPC networks~\cite{perello2013all}, some in combination with electrical solutions~\cite{Farrington2010Helios}\cite{Wang2010cthrough} (referred to as \textit{hybrid}) while others solely rely on optical components~\cite{Mellette2017RotorNet}\cite{Chen2014OSA} as the interconnect. \\
Helios~\cite{Farrington2010Helios} is a hybrid approach based on a 2-layer tree that combines conventional electrical packet switches with optical circuit switches based on MEMS. Although supplementing a traditional tree network with optical MEMS switches was shown to reduce power consumption and the number of switching elements, the circuit-switched approach of MEMS switches does not provide the flexibility and thus the performance of FlexLIONS. \\
Researchers at UC San Diego also proposed the use of optical circuit switching with their design RotorNet~\cite{Mellette2017RotorNet} which iterates over a fixed, static set of pre-determined optical port matchings to provide uniform random bandwidth across all endpoints. These rotations are done in a traffic agnostic manner to eliminate the need for a centralized control plane. While this approach overcomes some of the main challenges of circuit-switched solutions (i.e., only point-to-point connectivity), it fails to address needs of latency-sensitive HPC workloads. \\ 
OSA~\cite{Chen2014OSA} is another architecture taking advantage of MEMS switches and also makes use of wavelength selective switches to achieve dynamic topology reconfiguration by exploiting spatial and wavelength switching. In OSA, P out of N available ports in MEMS can be connected to each node. Parameter P defines the upper limit in number of nodes one node can connect to, but also affects the total number of nodes connected to one MEMS switch. Therefore, OSA comes with a trade-off between the total number of nodes and the amount of connectivity one system can achieve.\\
Rather than studying opportunities of integrating SiP switches into HPC networks, several studies have been dedicated to the design and implementation of the SiP switches itself. The most popular approach found in literature is the use of broadband ring resonators~\cite{DasMahapatra2014BroadMR}\cite{nikolova2017modular} which can be reconfigured to either drop or let pass the whole spectrum of wavelengths in a network from one waveguide to another, allowing to reconfigure arbitrary interconnections inside a switch. One  drawback is, as discussed earlier, the fact that BRRs are colorblind and do not allow fine-grained switching (and thereby limit the flexibility of a fabric for bandwidth reconfiguration). Besides, the number of BRRs required in an all-to-all switch grows quadratically with the number of ports, severely limiting its scalability. The main disadvantage of relying of a large number of BRRs, however, is that thermo-optic control for large numbers of BRRs is extremely challenging and impractical. \\
Finally, AWGRs-based data center switches have also been proposed in the scientific literature~\cite{proietti2018experimental}~\cite{cao2015hi}\cite{cao2015experimental}\cite{yin2013lions}\cite{proietti2013scalable} and generally seem to be very efficient in providing all-to-all switching capability with low crosstalk and loss. However, as described earlier, AWGRs alone can only provide the interconnection of nodes and, as a passive platform, cannot be reconfigured to adapt to varying communication demands. Nevertheless, FlexLIONS showed that the combination of AWGRs, MEMS, and MRs results in a powerful switch fabric exploiting the benefits of each SiP device. 